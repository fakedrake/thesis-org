\dictum[something]{ A dictum here }

\begin{summary}
\item Todo
\end{summary}

\begin{correction}{Related work}
  \emph{XXX: export from org notes}
\end{correction}

The fundametal problem being solved by FluiDB falls under the very
general umbrella of \emph{optimizing multiple queries simulatenously}. This
problem is central to database research and there is a very wide
diversity of approaches to solving it. It is useful, however, to
understand this less like a single problem more like a family of
problems. To help the reader understand the philosophy of FluiDB a
high level view we propose an organization of the proposed algorithms
that solve subsets of that problem exist in a space organized around
two poles:

\begin{itemize}
\item The algorithm solves a \emph{set} of queries simulatenously
\item The algorithm solves one query at a time from a \emph{stream} of queries.
\end{itemize}

While many design decisions are informed by the space a piece of work
occupies in relation to this dipole, virtually all approaches involve
identification of common subexpressions or sub-plans that can be
shared between different queries and avoiding duplicate
evaulation. This is indeed the unifying element of the work in the
field.

It is easier to conceptualize

The former problem has heavily researched for decades and is a major
 subfield in database research
 \cite{sellisMultipleQueryOptimization1987,royEfficientExtensibleAlgorithms2000,ComputingQueriesDerived}. It's
 focus is sligtly narrower as it focuses on identifying common
 subqueries in an unordered set of queries.

The latter has also been dubbed the View Selection Problem (VSP) but
research has mostly focused on static query workloads,
i.e. information about the query workload is known in advance and
views are pre-materialized
\cite{phanDynamicMaterializationQuery2008a}. Recently there have been
attempts to mitigate this based on reinforcement learning and genetic
algorithms due to high volumes, diversity and unpredictability of
queries in modern workloads.

Other systems:

\begin{itemize}
\item StagedDB \cite{harizopoulosStagedDBDesigningDatabase2005} and QPipe
\cite{harizopoulosQpipeSimultaneouslyPipelined2005} share between
queries that arrive within a window.
\item MonetDB \cite{idreosMonetdbTwoDecades2012} and CoScan
\cite{wangCoscanCooperativeScan2011} do dynaic scheduling to solve
queries together.
\item CJoin \cite{candeaScalablePredictableJoin2009} uses an "always-on"
plan of join operations to execute the joins of all concurrent
queries.
\item Crescando \cite{ssgiannikisCrescando2010} and IBM Blink
\cite{ramanConstanttimeQueryProcessing2008} attempt to share
individual table scans between queries to reduce IO.
\item DataPath \cite{arumugamDataPathSystemDatacentric2010}
\item SharedDB \cite{giannikisSharedDBKillingOne2012} and BatchDB
\cite{makreshanskiBatchDBEfficientIsolated2017} use a global query
plan for grouped queries.
\item MQJoin \cite{makreshanskiMqjoinEfficientShared2016a} shares join
plan for grouped queriesexecution between hundrends of queries.
\item Sloth \cite{cheungSlothBeingLazy2016} CrocodileDB
\cite{shangCrocodileDBEfficientDatabase2020} exploit lazy or deferred
execution to increaase the opportunitites for batching and therefore
sharing.
\item DynaMat \cite{kotidisDynamatDynamicView1999} and watchman
\cite{scheuermannWatchmanDataWarehouse1996} do dynamic view
selection.
\end{itemize}

ShareDB, DataPath, CJoin and QPipe focuses on sharing during runtime

the pay-one-get-hundrends-for-free do only SQL rewriting to do runtime
  plan for grouped queriessharing.


\cite{mounaProResProactiveReselection2022}


\section{Query equality}
\label{sec:orga919a5f}

\section{Subquery selection algorithm}
\label{sec:org70319fb}

\section{Eviction policy}
\label{sec:orgd16def2}

%%% Local Variables:
%%% mode: latex
%%% TeX-master: t
%%% End:
