\dictum[Vamvakaris]{Don't hit hard the beads, work makes a man.}

\begin{summary}
\item Tables are stored in a memory file systems and records are
  stored as POD binary objects organized in 4K pages.
\item The extent of each relation involved in a plan is converted to
  C++ struct and each predicate is converted to a C++ callable
  class.
\item These parameterize the templatized operators to allow the
  compiler to generate highly specifialized code.
\item In this chapter we discuss the implementation important
  operations and their reverse.
\end{summary}

The logical plan is translated into a physical plan that has the form
of a c++ program. The motivation for using code generation relates to
expression level optimizations so it’s less work on some fronts
(although more work in others). Also gdb for debugging. Remember that
compilation time is not pure overhead, we would have to duplicate at
least some low level optimizations in the haskell.

\section{Codegen introduction}

Code generation is becoming a more and more common in RDBMSs. Used
mainly in in-memory databases, where disk IO does not dominate the
runtime, it aims to minimize the overhead of data access function
calls, to optimize the operdicates and numerical expressions and to
avoid indirection in tight loops. Approaches to code generation fall
generally on a spectrum between two extremes:

\begin{itemize}
\item Transpilation of the physical plan to a low level programming
  language like C or C++ for every query
\item JIT compilation of small parts of the plan as the query
  executes.
\end{itemize}

FluiDB follows the approach of Krikellas
et. al. \cite{krikellasGeneratingCodeHolistic2010} falling far to the
former end of the spectrum. We generate very specific, template-heavy
C++ code for every query and we call to an off-the-shelf compiler to
generate highly optimized machine code.

\section{Storage memory management}

We opted for delegating the task to the OS and use use use the tmpfs
as a storage layer to our database. The tmpfs filesystem depends on
the the \emph{shmem} module (as of linux v5.13) for handling file
operations. shmem is a resizable virtual memory filesystem for
linux. Where a typical persistent filesystem stores files in a block
device and caches pages in memory for efficiency, shmem keeps files
exclusively as pages in the page cache. The OS tries to keep all pages
in memory but when resources start running out it writes pages in
swap.

Assuming that the pages are not in swap, normal reads for shmemfs
using \cpp{read()} are equivalent to copying pages from the page cache to
the user space. \cpp{shmem} writes on the other hand operate directly to
the pages. We can mitigate the copying overhead of reads using \cpp{mmap}
which will remap the page to the address space of the application.

The problem with this approach is that, while it allows us to minimize
copying, we still need to run system calls in tight loops, which can
be very computationally expensive. This can't be completely mitigated
unless we move the "storage" layer to the userspace, re-implementing
the memory management that we get for "free", in terms of engineering
effort, from the shmem module so we leave that for a future version of
FluiDB.

On the other hand this approach allows FluiDB to easily be adapted to
operate over any filesystem backed by different storage technologies
like non-volatile memories.

\section{Data layout}

Before we get into the details of the actual physical plan (in the
form of C++ code) we need to state some assumptions about the layout
of the code that is assumed by the primary data.

FluiDB is a \emph{row store system} and it depends on the filesystem
for itnermediate result lookup and page management. For code
generation to make sense the file system is, in particular, a tmpfs
filesistem that resides entirely in ram. This is not an ideal solution
performance-wise and in the future we plan on using a less OS-reliant
way of managing storage resources but it is good enough for now. In
particular we use one file per table or intermediate result, and each
file is simply a sequence of recods organized into pages. For now
FluiDB does not support any kind of indexing or compression. FluiDB
can, of course, be run over any filesystem but non-memory based file
systems diminish benefit of code generation, making performance IO
bound.

With this in mind there are three parts to understanding the
principles of FluiDB storage:

\begin{itemize}
\item The format in which primary data is inserted into the database.
\item The layout of the data withing the database
\item The transformation from the former to the latter.
\end{itemize}

\subsection{Initial data conversion}

Starting with the initial data, as it stands, for FluiDB to be adapted
to a particular dataset it requires the primary data in CSV format and
some haskell code describing the shape of the data and the database
configuration. Our experiments so far have been revolving around the
SSB TPC-H benchmark so the format expected is the plaintext format
that dbgen \cite{perivolaropoulosFakedrakeSsbdbgen2021}
generates. This is comprized by two steps: first a haskell program
parses the CSV records into standard-layout binary objects that can be
directly cast to C/C++ standard-layout structs. These binary objects
are stored one after the other in a flat binary file. The end result
of what we want to do to be able to execute code similar to the one
presented in listing \ref{lst:bama_to_dat}.

\begin{code}
\begin{cppcode}
template<typename R, size_t batch_size=2000>
void bama_to_dat(const std::string& bama_file, const std::string& dat_file) {
  int fd;
  size_t read_bytes;
  std::array<R, batch_size> batch;
  // Open the file in binary mode
  fd = ::open(bama_file.c_str(), O_RDONLY);
  // The writer is the object used by all the libraries and it will
  // write each record in pages.
  Writer<R> w(dat_file);
  do {
    // Read a batch of data,
    read_bytes = ::read(fd, batch.data(), sizeof(batch));
    // Use the writer object to write each record.
    for (size_t i = 0; i < read_bytes / sizeof(R); i++) {
      w.write(batch[i]);
    }
    // Keep reading until a batch is cut short.
  } while (read_bytes == sizeof(batch));
  // Wrap up.
  w.close();
  close(fd);
}
\end{cppcode}
  \caption{\label{lst:bama_to_dat}For standard FFI communication C++
    structs that do not contain fancy constructors}
\end{code}

For this to work we refer to the C++ standard
\cite{14:00-17:00ISOIEC14882}. The types that represent table rows must
be standard layout. According to the standard:

\begin{quote}
A class S is a standard-layout class if it:
\begin{itemize}
\item has no non-static data members of type non-standard-layout class (or
array of such types) or reference,
\item has no virtual functions and no virtual base classes
\item has the same access control for all non-static data members,
\item has no non-standard-layout base classes,
\item has at most one base class subobject of any given type,
\item has all non-static data members and bit-fields in the class and its
base classes first declared in the same class, and
\item has no element of the set M(S) of types as a base class, where for
any type X, M(X) is defined as follows. [Note: M(X) is the set of
the types of all non-base-class subobjects that may be at a zero
offset in X.]
\begin{itemize}
\item If X is a non-union class type with no (possibly inherited)
non-static data members, the set M(X) is empty.
\item If X is a non-union class type with a non-static data member of
type X\textsubscript{0} that is either of zero size or is the first non-static
data member of X (where said member may be an anonymous union),
the set M(X) consists of X\textsubscript{0} and the elements of M(X\textsubscript{0}).
\item If X is a union type, the set M(X) is the union of all M(Ui) and
the set containing all U\textsubscript{i} , where each U\textsubscript{i} is the type of the i
th non-static data member of X.
\item If X is an array type with element type X\textsubscript{e}, the set M(X) consists
of X\textsubscript{e} and the elements of M(Xe).
\item If X is a non-class, non-array type, the set M(X) is empty.
\end{itemize}
\end{itemize}
\end{quote}

The records we generate certainly conform to these requirements. An
example is the supplier row in \ref{lst:recodr_class}. Therefore the
object is trivially copyable and occupies contiguous bytes of
storage. This means that we can safely write each record \cpp{R} as
\cpp{sizeof(R)} contiguous binary data to a file and expect to find the
same value of \cpp{R} when we read it. Based on this we can safely copy
binary record objects from memory to disc and visa versa.

\begin{code}
\begin{cppcode}
class Record {
public:
  Record(unsigned __s__suppkey, fluidb_string<25> __s__name,
         fluidb_string<40> __s__address, fluidb_string<16> __s__city,
         fluidb_string<16> __s__nation, fluidb_string<13> __s__region,
         fluidb_string<15> __s__phone)
    : s__suppkey(__s__suppkey),
      s__name(__s__name),
      s__address(__s__address),
      s__city(__s__city),
      s__nation(__s__nation),
      s__region(__s__region),
      s__phone(__s__phone) {}
  Record() {}
  std::string show() const {
    std::stringstream o;
    o << s__suppkey << " | " << arrToString(s__name) << " | "
      << arrToString(s__address) << " | " << arrToString(s__city) << " | "
      << arrToString(s__nation) << " | " << arrToString(s__region) << " | "
      << arrToString(s__phone);
    return o.str();
  }
  bool operator==(const Record& otherRec) const {
    // compare each field ...
  }
  bool operator!=(const Record& otherRec) const {
    // compare each field ...
  }
  unsigned s__suppkey;
  fluidb_string<25> s__name;
  fluidb_string<40> s__address;
  fluidb_string<16> s__city;
  fluidb_string<16> s__nation;
  fluidb_string<13> s__region;
  fluidb_string<15> s__phone;
};
\end{cppcode}
  \caption{\label{lst:recodr_class}The supplier row representation in
    the generated C++ code. The \cpp{fluidb_string} type is a
    constant size arraw of characters.}
\end{code}

However, in the case of parsing we need to take great care with byte
alignment which is compiler dependent. Fortunately Clang and GCC
informally aggree on the algorithm for \cpp{alignof(<cls>, <member>)} for
standard layout objects. The algorithm is presented in
\ref{lst:record_byte_padding}

\begin{code}
\begin{haskellcode}
schemaPostPaddings :: [CppType] -> Maybe [Int]
schemaPostPaddings [] = Just []
schemaPostPaddings [_] = Just [0]
schemaPostPaddings schema = do
  elemSizes <- sequenceA [cppTypeSize t | t <- schema]
  spaceAligns' <- sequenceA [cppTypeAlignment t | t <- schema]
  let (_:spaceAligns) = spaceAligns' ++ [maximum spaceAligns']
  let offsets = 0 : zipWith3 getOffset spaceAligns offsets elemSizes
  return $ zipWith (-) (zipWith (-) (tail offsets) offsets) elemSizes
  where
    getOffset nextAlig off size =
      (size + off)
      + ((nextAlig - ((size + off) `mod` nextAlig)) `mod` nextAlig)
\end{haskellcode}
  \caption{\label{lst:record_byte_padding}Algorithm to infer the
    padding of members according to the Itanium ABI.}
\end{code}


Once the bama files are generated C++ code is generated for parsing
the file calls into the C++ function \cpp{bama_to_dat} that is parametric
to the type of the object being and uses the BAMA library to write
objects, thus making sure that the data is readable by the
operators. \cpp{bama_to_dat} simply reads the input \texttt{.bama} file as a
stram of records of size \cpp{sizeof(Record)} casting the bytes with
\cpp{reinterpret_cast<Record*>} into the record. It then use the bame
record writing facility \cpp{Writer<R>::write} that takes care of
organizing the record into pages. Thus the final \emph{data file} is
created that is ready for use by the generated code.

\begin{code}
\begin{cppcode}
#include <bamify.hh>
class Record {
  // ...
};
int main(int argc, char* argv[]) {
  bama_to_dat<Record>("supplier.bama","supplier.dat");
}
\end{cppcode}
  \caption{Convert a bama file to a paged data file.}
\end{code}

\subsection{Pages}

We translate the flat binary files into \texttt{.dat} files tath
contain the table daata in its final format. The basic block of the
file is the \cpp{Page}, and the file is simply a raw squence of
pages. A page is constant-size data structure that contains up to
\(\left\lfloor\frac{S_{rec}}{S_{pg}} \right\rfloor\) \emph{whole}
records where \(S_{rec}\) is \cpp{sizeof(Record)} and \(S_{pg}\) is
the size of the page which typically is 4KB. All pages must contain as
many whole records as can fit except the last one.

All transactions with the storage are made at the page level: we
either read or write only entire pages. These operations are
abstracted by the \cpp{Reader} and \cpp{Writer} classes. We use one more level
of abstraction for convenience, the higher order \cpp{eachRecord}
function. To demonstrate what the interface looks like we present the
implementation of \cpp{eachRecord} \ref{lst:each_record}.

\begin{code}
\begin{cppcode}
// Fn could be instantiated to std::function<void(const R&)> but that
// will *always* forbid f from being inlined.
template <typename R,typename Fn>
inline void eachRecord(const std::string& inpFile,Fn f) {
  Reader<R> reader;
  size_t i = 0;
  reader.open(inpFile);
  while (reader.hasNext()) {
    i++;
    f(reader.nextRecord());
  }
  reader.close();
}
\end{cppcode}
\caption{\label{lst:each_record}}
\end{code}

As alluded to in the previous section both the \cpp{Writer} and \cpp{Reader}
use \cpp{reinterpret_cast} to "serialize" and "deserialize" the data
respectively.

\subsection{C++ row iterators}

It is important for the planner to have full control of the storage
budget and assume that no significant memory is required for
evaluating each operator. For that reason we require that all
operators' algorithms have constant space complexity. This may mean
major compromizes for some algorithms with respect to the time
complexity like join and aggregation. Fortunately we can mitigate that
by taking advantage of the set semantics of FluiDB relational algebra
and sorting input tables in-place before running joins or
aggregations. Our particular implementation of in-place sorting hinges
on the \cpp{RecordMap} type that provides C++ random access iterators
iterators to the records of a file, abstracting the page reads and
writes. We pass these iterators to \cpp{std::sort} (see listing
\ref{lst:record_map_sort}) which runs insertion sort for small ranges
to take advantage of the processors reorder buffer and quicksort for
longer ones.

\begin{code}
\begin{cppcode}
RecordMap<size_t> fs("/tmp/removeme.dat");
std::sort(fs.begin(), fs.end());
\end{cppcode}
  \caption{\label{lst:record_map_sort}Using a \cpp{RecordMap} to sort
    the records of a file.}
\end{code}

The operation of \cpp{RecordMap} is very simple. Each iterator is paired
with the page that contains the record it points to, when an iterator
is the only iterator pointing inside a page and is incremented or
decremented and no longer points to the same page the old page is
written back to the file and the new page is read. When the only
iterator pointing to a page is deleted the page is written to the
file. More than one iterators pointing to the same page do not
maintain different copies of that page and the last one to be
destroyed or to leave the page triggers the page to be written back to
the file.
\section{Physical planning}

The fundamentall logic of the code generator is fairly simple: The
input of the code generator is a a list of transitions generated by
the FluiDB planner. The reader is reminded that each transition has
one of three kinds:

\begin{itemize}
\item Trigger of a t-node. The input n-nodes at the time of the trigger
are materialized and the trigger itself materializes a subset of the
output n-nodes
\item Reverse trigger of a t-node, which represents the
materialization of a subset of the input n-nodes from materialized
output nodes
\item The deletion of an n-node.
\end{itemize}

Transitions are meant to be executed in the order they appear in the
received sequence to preserve correctness. However there is no
one-to-one correspondence between operators and transitions. Instead,
as we saw in secion \ref{sec:cluster_internals}, each operation
corresponds to a cluster of connected t-nodes. The first step of the
code generator, therefore, is to group the low level transitions
received by the planner into higher level batches that correspond to
exactly one operation. This process is driven by intermediate n-nodes,
ie helper nodes that do not correspond to materializable relations,
but are rather part of the graph to reify the valid combinations that
the planner is allowed to materialize. The constraint we are trying to
preserve while batching the low level transitions such that the
cluster level transitions do not materialize any intermediate nodes.

Since intermediate nodes are always internal to clusters and no
cluster shares a t-node with another cluster, each batch of
transitions corresponds to exactly one cluster, except for deletion
transitions which are standalone. Furthermore, each cluster
corresponds to exactly one relational operator, which we also include
in the higher order transition.

\begin{comment}
\subsection{AST to literal code}

In fludb the C++ AST is converted to a literal code string. This is
done in two steps:

\begin{itemize}
\item Translation of the AST into fragments of soft literal code, that
is code in a format similar to that of a string, which has the
property to maintain the uniqueness of symbols under concatenation.
\item Concatenation of soft code fragments and translation to a literal
string representing the C++ code.
\end{itemize}

This is done by the \hask{Codegen} typeclass. Any type implementing
Codegen can be transformed to an IsCode type. IsCode is any type that
can be directly converted into a string of C++ code. For now consider
that IsCode only refers to normal strings. In this section we will
look at the transformation of a valid AST into code.

  \subsubsection{Code hygene}

  A major concern in any transpiler, or code generation system generally,
  is the generation of unique symbols. The problem is similar to the one
  regarding scheme’s hygienic macros, although the problems for those
  are much more complex than our particular case. This is mostly because
  our plans expand to a different language and expansion happens exactly
  once. Consider the following C++ AST that we want to transform into
  code.

  \[
    R_1 \Join_{\theta} R_2
  \]

  The most naive and straightforward way is to convert it directly into
  a C++ string generating symbols deterministically based on the symbols
  used in the query.

\begin{cppcode}
struct R1 {...};
struct R2 {...};

bool theta (R1 r1, R2 r2) {
  ...
}

int main () {
  ...
  auto op = Join<R1,R2>(theta,"R1.dat", "R2.dat");
  op.run();
}
\end{cppcode}


  This approach works for a narrow class of cases where the names
  \cpp{theta}, \cpp{R1}, and \cpp{R2} do not collide with any other
  names in the program and where they are referred to anywhere outside
  of the scope of the particular block. The obvious approach would be
  the Common Lisp approach to ``hygene'' where we generate a new symbol
  for each identifier. The naive and obviously wrong way to do that
  would be

\begin{cppcode}
struct R1_1 {...};
struct R2_2 {...};

bool theta_3 (R1_4 r1, R2_5 r2) {
  ...
}

int main () {
  ...
  auto op = Join<R1_6,R2_7>(theta_8,"R1.dat", "R2.dat");
  op.run();
}
\end{cppcode}

  This goes too far disconnecting the names used in the declarations
  from the symbols themselves. This is fixable by discriminating between
  declarations:


\begin{cppcode}
struct R1_1 {...};
struct R2_2 {...};

bool theta_3 (R1_4 r1, R2_5 r2) {
  ...
}

int main () {
  ...
  auto op = Join<R1_1,R2_2>(theta_3,"R1.dat", "R2.dat");
  op.run();
}
\end{cppcode}

  New problems arise if we want to concatenate different code
  snippets.

  \begin{code}
\begin{haskellcode}
-- todo
\end{haskellcode}
    \caption{Variable reuse}
  \end{code}

  If these generated code snippets uniquify their variables separately
  the variable name a will be re-declared making the compiler
  unhappy. We resolved this issue by introducing the SoftUNameCode
  functor. SoftUNameCode a implements the IsCode interface as long as a
  implements it as well.

  \begin{code}
\begin{haskellcode}
data SoftUNameCode a = SoftUNameCode {
  uCodeTail :: Maybe (UniqueSymbol,SoftUNameCode a),
  uCodeHead :: a,
  uCodeMaxId :: UId
  }

data UniueSymbol a = UniqueSymbol {
  symbolUId :: UId,
  symbolMkName :: UId -> a
  }
\end{haskellcode}
    \caption{Variable reuse}
  \end{code}

  What this amounts to is a list of code objects interleaved with unique
  symbol names. Each unique symbol name is a combination of an unique id
  and a function that would generate a unique string of code based on
  that id. This unique id must be incrementable.

  When two \hask{SoftUNameCode} objects are concatenated, in the simple case
  we aim for the variable names to be completely disjoint. This is
  achieved by simply incrementing all symbolUIds in the right hand side
  of the concatenation operator by the left hand side uCodeMaxId.

  In the slightly more complex case where we know that the same unique
  symbol is both on the left and the right hand side of the
  concatenation operator we use co-join \hask{SoftUNameCode} a into
  \hask{SoftUNameCode (SoftUNameCode a)} pushing the shared symbol in the
  internal layer. When concatenating SoftUNameCode, only the top layer
  of the right hand side is changed. After concatenation the functor is
  flattened:

  With this method we can flexibly concatenate code keeping certain
  variables hygienic while maintaining reference to others.
\end{comment}

\section{Operator implementations}


The C++ AST is expressed as a tree of haskell algebraic data
types. They do not capture the entire C++ language, only the following
concepts concepts:

\begin{itemize}
\item Function declaration
\item Function application and arguments
\item Expressions
\item Code symbols
\item Assignment
\item Includes
\item Literals
\item Classes
\item Member clarations (for record printing)
\item Algorithm selection
\item Global declarations
\end{itemize}

The code generated version of each operator is parameterized by the
record types of its inputs and outputs and highly specific code is
generated by the C++ templating system.

All operator implementations live in the BAMA library which branched
out from \cite{krikellasGeneratingCodeHolistic2010}. The BAMA library
makes heavy use of templates and constexpr to generate very
query-specific machine code for each operator.

\subsection{Generated code structure}

The FluiDB code generator generates a couple of different kinds of C++
components to construct the final file. In this section we will
discuss the kinds of C++ structs that FluiDB generates in order to
parameterize the BAMA operators at compile time, leveraging the C++
template system.

In the following subsections when we say \emph{compile time} we refer ti
the compile time of the generated code, not compile time of FluiDB.

\subsubsection{Maybe types}

Maybe types are used to indicate optional outputs of operators. For
example an operator \(\sigma_p\) may yield \(\sigma_p A\) or
\(\sigma_{\neg p} A\) or both. On one hand the precise outputs that
are required are important for performance and memory budget
management, and on the other any combination of these outputs can and
should be generated through one pass over the input. Furthermore
information about which outputs are required is known at compile time
so the generated code should be specific to the combination of outputs
that are required.

To address this, instead of passing in a simple \cpp{std::string} as a
path, we pass one of two types \cpp{Nothing} or \cpp{Jusr} (listing
\ref{lst:maybe_type_cpp}) that wrap the file path. Both these types
contain an \cpp{isNothing} static constexpr type that can be used by
\cpp{if constexpr (...)} expressions in the selection operator to help
the compiler generate highly specialized machine code.

\begin{code}
\begin{cppcode}
template <typename T>
struct Just {
  Just(T t) : value(t) {}
  T value;
  constexpr bool operator==(Just<T> const& j) const {
    return this->value == j.value;
  }
  static constexpr bool isNothing = false;
};
template <typename T=std::string>
struct Nothing {
  Nothing(T s) {}
  Nothing() {}
  static constexpr bool isNothing = true;
};
\end{cppcode}
\caption{\label{lst:maybe_type_cpp}The type level maybe}
\end{code}

\subsubsection{Record types}

FluiDB is a row store engine and handles table rows by generating
record types specific to each table, where objects of that type are
rows of the corresponding table. We went over some aspects of the the
genrated record types when discussing the conversion of primary data
to FluiDB specific binary data. Records types directly or indirectly
parameterize all operators. A selection operator \(\sigma_p\) instance
must be parameterized by the record type of its input which will be
the same as the output. A projection \(\pi\) is parameterized by the
records of both the input and the output type. A generated record type
must must have some structural properties to function properly with
BAMA and the rest of the generated types:

\begin{itemize}
\item First and foremost it must be standard-layout type provide so it can
be trivially copied from and to binary blob files. This means that
there should be no virtual members, the destructor must be trivial
and no static members. We further impose the constraint that they
must not have a base class to avoid some more intricate constraints
imposed by the C++ standard on standard-layout classes.
\item it must provide non-uniquified names for each field of the row it
represents so other generated types (like predicates and subtuple
extraction types) have direct access to the underlying data.
\item They should be comparable by equality (\cpp{==}), inequality (\cpp{!=}) and
they should be orderable \cpp{<}. The actual semantics of the ordering
are not important as long as there is a deterministic way of
ordering records of the same type.
\item They should be hashable. We get that for free since they are
standard layout objects but it is an important property that is
useful for equijoins as we will see in the section about equi-joins.
\item Finally every record type must implement an \cpp{std::string
show()} function that serialized the contents into a human readable
string.
\end{itemize}

\subsubsection{Predicate types}

Predicate types have the function of n-ary boolean functions \(A_1
\times ... \times A_k \to \{T,F\}\) useful for selections and
\(\theta\)-joins so in practice they are unary or binary. They are
passed as template arguments to the operators so the compiler has the
chance to inline them to avoid a function call. To avoid too much code
clutter they are expected to provide the types of the function domain
(demonstrated in listing \ref{lst:pred_type_cpp}) in order to minimize
the number of template parameters and keep the generated code
relatively human readable.

\begin{code}
\begin{cppcode}
class Predicate3421 {
  typedef Record123 Domain0;
  typedef Record32 Domain1;
  bool operator() (const Domain0& rl, const Domain1& rr) {
    return rl.__field1 < rr.__field2;
  }
}
\end{cppcode}
\caption{\label{lst:pred_type_cpp}The shape of a generated predicate type.}
\end{code}

\subsubsection{Record transformation types}

Record transformations, much like predicates are callables
representing pure functions, only the codomain is a record type,
instead of a boolean. They are used by joins to combine the matching
records, by projections and aggregations to produce new records from
the inputs, by equi-joins to extract the subtuples to be checked for
equality, etc. Much like predicates they can be queried for their
domain and codomain to reduce the number of template arguments in
operators (see listing \ref{ref:transform_type})

\begin{code}
\begin{cppcode}
class Transform {
  typedef Record123 Domain0;
  typedef Record32 Domain1;
  typedef Record10 Codomain;
  Codomain operator() (const Domain0& l, const Domain1& r) {
    return Record10(l.__key,l.__field1, r.__key, r.__field2);
  }
};
\end{cppcode}
  \caption{\label{ref:transform_type}A record transformation type
    defines objects with no internal state that are callable.}
\end{code}


\subsubsection{Operators}

Operators are not generated but they are parameterlized by all the
kinds of gernerated code we mentioned. They are clases that are
constructed using maybe-filenames, record transformers, record types,
and predicates and implement the \cpp{run()} method which actually
runs the internal code. For demonstration purposes an abbreviated
version of the \(\sigma_p\) operator, which highlingts the shape of an
operator class is presented in listing \ref{lst:sel_operator_cpp}.

\begin{code}
\begin{cppcode}
template <typename Predicate,
          typename PrimaryOutType,   // Maybe(std::string)
          typename SecondaryOutType  // Maybe(std::string)
          >
class Select {
  typedef typename Predicate::Domain0 Record;

public:
  Select(PrimaryOutType prim, SecondaryOutType sec, std::string in)
    : primary_file(prim), secondary_file(sec), infile(in) {
    static_assert(!PrimaryOutType::isNothing || !SecondaryOutType::isNothing,
                  "Both primary and secondary output files are Nothing.");
  }

  ~Select() {}

  void run() {
    // ...
  }

  void print_output(size_t x) {
    // ...
  }

private:
  PrimaryOutType primary_file;
  SecondaryOutType secondary_file;
  std::string infile;
  static Predicate predicate;
};
\end{cppcode}
  \caption{\label{lst:sel_operator_cpp}The selection operator. It is
    parameterized by the predicate and the primary and secondary
    output types. Enough information about these values is known at
    compile time such that the compiler can generate highly
    speclalized code.}
\end{code}

Unfortunately, even recent iterations of the C++ standard do not
include template type inference for classes and structs, therefore, to
make the generated code simpler we wrap the constructor into a
function like shown in listing \ref{lst:sel_function_cpp}. This allows the
generated code for selection to look like the one presented in listing
\ref{lst:gen_out_code}.

\begin{code}
\begin{cppcode}
// C++17 can only infer typenames (primaryout secondaryout) in
// function templates.
template<typename Predicate,
         typename PrimaryOutType,   // Maybe(std::string)
         typename SecondaryOutType  // Maybe(std::string)
         >
auto mkSelect (const PrimaryOutType prim,
               const SecondaryOutType sec,
               const std::string& in) {
  return Select<Predicate,
                PrimaryOutType,   // Maybe(std::string)
                SecondaryOutType> // Maybe(std::string)
    (prim, sec, in);
}
\end{cppcode}
\caption{\label{lst:sel_function_cpp}The C++ declaration of the select.}
\end{code}


\begin{code}
\begin{cppcode}
int main() {
  // ...
  {
    auto op = mkSelect<Predicate32>(Just("data123.dat"), Just("data53.dat"),
                                    "lineitem.dat");
    op.run();
    op.print_output();
  }
  // ...
  return 0;
}
\end{cppcode}
  \caption{\label{lst:gen_out_code}A block representing a particular
    operator.}
\end{code}


\subsection{Select algorithm}

The selection algorithm is likely the simplest of the implemented ones
because FluiDB does not support indexes and makes no assumptions about
the ordering of the data. In the forward variety it is implemented as
either a partition or a selection depending on which of the outputs it
is materializing. In its backward variety it is essentially a union
\(A \equiv \sigma_p A \cup \sigma_{\neg p} A\). As we discussed
already (listing \ref{lst:sel_function_cpp}) the BAMA library expects
only a predicate class as a template parameter and three filnames as
its arguments.

It is important to remind the reader that no order is expected to be
preserved by union, which is to say that the transition \(A
\Rightarrow \{\sigma_p A, \sigma_{\neg p} A\} \Rightarrow A\) does not
preserve the order of the nodes. The reader is also reminded that
FluiDB does not support \hask{NULL} values to preserve the correctness of
the reverse select operation.


\subsection{Projection algorithm}
\label{sec:projection_algorithm}

The implementation of the projection algorithm is parameterized by two
template arguments that extract complementary subtuples from an input
relation, affording each one with a common unique subtuple to
facilitate the inverse operation. The inverse of a projection then is
simply an equijoin between the two produced slices based in that
shared unique subtuple.

While the BAMA side of the calculation is fairly straigtforward, on
the Haskell side (code generation) it is slightly more complex. The
first piece of the puzzle is the representation of the
projection. When the query is initially processed so that projections
are augmented to expose unique subtuple we actually change the
represetation of projections from \hask{QProj [(e,Expr e)]} to
\hask{QProjI [e] [e] [(e,Expr e)]}. The two extra parameters typed
\hask{[e]} represent the complement of the projection (the columns
from the input not exported by the projection) and a unique subtuple
that the complement and the projection must have in common, while
\hask{[(e,Expr e)]} is the normal projection. We make this
transformation because the easiest and cheapest place to calculate the
first and second parameters (complement and correspondence) is during
the projection augmentation phase that happens immediately after
parsing. Therefore we have all projections carry information about the
complement at the level of the RA operator.

In particular this process is combined with remapping of uniuque
subtuples into the projection. For example, during the unique column
exposure process the query \sql{select p_color from part} is found not
to be valid because \sql{p_color} is not unique for each row. For that
reason, during preprocessing we remap the \sql{p_partkey} column since
\sql{p_partkey} is the only unique column of \sql{part} this way the
constraint that every valid relation in FluiDB needs to have at least
one subtuple that is unique within that relationis satisfied and the
query becomes

\begin{sqlcode}
select p_partkey,p_color from part
\end{sqlcode}

In order to be able to reconstruct the input we require that at least
one unique subtuple is shared between the primary projection relation
and the complement. In the example demonstrated this means the
\sql{p_partkey} column is duplicated in both output tables of the
operator. The particular valid projection operator that would be
generated the \hask{QProj} is presented in listing
\ref{lst:real_proj_operator}.

\begin{code}
\begin{haskellcode}
op =
  QProjI
    -- All keys except p_color
    [p_partkey,p_name,p_mfgr,p_category,p_brand1,p_type,p_size,p_container]
    -- The unique subtuple (just one column)
    [p_partkey]
    -- The the actual projection
    [(p_partkey,E0 p_partkey),(p_color,E0 p_color)]
\end{haskellcode}
  \caption{\label{lst:real_proj_operator}The projection operator
    produced from the SQL query \sql{select p_partkey,p_color from
      part}.}
\end{code}

The algorithm for reverse projection is almost a standard join. The
only difference is that in our semantics of join, the relation \(A
\Join_{a = b} B\) contains both the column \(a\) and the column \(b\),
while when joining a projection with it's complement we only create a
single copy of the shared subtuple. We will see more details about
join in the next section.


\subsection{The join operator}

We distinguish between two kinds of join operators: the equijoin and
the general \(\theta\)-join. The former is specific to cases where
subtuples from each table are compared and the implementation is
simply a merge join, while the latter is used for general predicates
and is essentially a piplelined \(\sigma(A \times B)\). The important
detail worth noting about join algorithm implementations in FluiDB is
that they must be able to produce antijoin relations \(A \lnsemi B\),
\(A \rnsemi B\) simultaneously with the join relation \(A \Join B\).

It is important for join to be space efficient, and particularly
constant-space, due to the assumption made by the planner that it has
control over memory. For general \(\theta\)-joins we use nested loop
joins. For equijoins we take advantage of the assumption that the
tables are unordered and use merge join after sorting the input tables
in-place. A sketch of the algorithm is presented in listing
\ref{lst:join_algorithm}

\begin{code}
\begin{pycode}
# This is a pythonic way of detecting which input stream finished is.
class LeftFinished:
    pass
class RightFinished:
    pass


def lnext(it):
    """If iteration finishes throw LeftFinished"""
    try:
        return next(x)
    except:
        raise LeftFinished

def rnext(it):
    """If iteration finishes throw RightFinished"""
    try:
        return next(x)
    except:
        raise RightFinished

@template(extractl,extractr,combine)
def merge_equijoin(in_l,in_r,anti_l,out,anti_r):
    # In place sort each of the inputs
    sort(in_l,key=extractl)
    sort(out_l,key=extractr)
    # Define iterators and current values for each of the inputs.
    it_l = iter(in_l)
    val_l = next(it_l)
    it_r = iter(in_r)
    val_r = next(it_r)

    try:
        # The actual merge join algorithm
        while True:
            # Gather blocks of equal records from left and right and write
            # out their product.
            if extractl(val_l) == extractr(val_r):
                # Gather a sequence of equal-key values from the left
                val_ls = []
                tmp = val_l

                while extractl(val_l) == extractl(tmp):
                    val_ls.append(val_l)
                    val_l = lnext(it_l,None)

                # Gather a sequence of equal-key values from the right
                tmp = val_r
                val_rs = []
                while extractl(val_r) == extractl(tmp):
                    val_rs.append(val_r)
                    val_r = rnext(it_r)

                # Write out their product.
                for l,r in product(val_ls,val_rs):
                    out.write(combine(l,r))

            # Push the non-equal records to the antijoins
            while extractl(val_l) < extractr(val_r):
                anti_l.write(val_r)
                val_r = rnext(it_r)
            while extractl(val_r) < extractr(val_l):
                anti_r.write(val_l)
                val_l = lnext(it_l)
    except LeftFinished:
        for l in it_l:
            left_anti.write(r)
    except RightFinished:
        for r in it_r:
            right_anti.write(r)
\end{pycode}
\caption{\label{lst:join_algorithm}The equi-join algorithm first sorts
  in place the inputs w.r.t. the equal subtuples and then merges
  them.}
\end{code}


The type of the join operator is defined in terms of a template such
that it can be specialized at compile time (see listing
\ref{lst:join_decl}). The template arguments are the following:

\begin{itemize}
\item Three file paths are provided, \cpp{OutFile}, \cpp{LeftAnti} and
  \cpp{RightAnti} are each either of type \cpp{Nothing} indicating to
  the compiler not to generate any code that relates to the particular
  output or of type \cpp{Just<std::string>} indicating to the compiler
  the opposite.
\item The subtuple extraction functions \cpp{LeftExtract} and
  \cpp{RightExtract}. These are record transformation functions that
  extract the subtuple that needs to
\end{itemize}

\begin{code}
\begin{cppcode}
template <typename LeftExtract, typename RightExtract, typename Combine,
          typename OutFile,    // Maybe(std::string)
          typename LeftAnti,   // Maybe(std::string)
          typename RightAnti>  // Maybe(std::string)
class Join;
\end{cppcode}
\caption{\label{lst:join_decl}Class declaration of the join operator}
\end{code}

The reverse join operator is composed of two pipelined steps:

\begin{itemize}
\item A projection and deduplication based on the unique subtuple of
the input on the join output to get a left or right semijoin
\item A union of the semijoin with the antijoin to get the input table.
\end{itemize}

As with the selection operator, the correctness of this reverse
operation is predicated on the fact that FluiDB has no notion of
\sql{NULL} values.


\subsection{Aggregation and sort algorithms}

The main challenge with implementing aggregations in the context of
FluiDB is maintaining constant space during its evaluation, as the
planner tries to use as much of the memory budget as possible to
maintain intermediate results. Therefore we opted against using an
auxiliary hash map used to group records. Like we did to implement
joins we perform aggregation in two steps: first we sort the records
in place based on the grouping columns and then aggregate in a single
pass.

\section{Conclusion}

In this chapter we discussed some physical attributes of how FluiDB
evaluates the final plans it constructs via code generation. We
discussed the storage layout and the fundamental ideas the code
generator was built around in order to generate highly specialized
code. Finally, we discussed some specific algorithms used by the BAMA
library that is called by the generated code to implement the actual
operators.

We further discussed the some caveats that relate to the FluiDB's
approach to code generation, namely that highly the specific code
generated can be expensive to compile from scratch and that the shmem
in-memory filesystem provided by linux is a cheap solution in terms of
engineering effort but would be better replaced with a solution that
does not involve system calls.
