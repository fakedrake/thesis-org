%% Start the page numbering in roman
\setcounter{page}{1}%

%% Make the title page
\maketitle%

\begin{precontent}
\begin{greek}%
\begin{normalsize}%
Έχω φυλάξει κάτι αποκόμματα με κάποιον\\%
που λέγανε πως είσαι συ.\\%
Ξέρω πως λένε ψέματα οι εφημερίδες,\\%
γιατί γράψανε πως σου ρίξανε στα πόδια.\\%
``Ξέρω πως ποτέ δε σημαδεύουνε στα πόδια.\\%
Στο μυαλό είναι ο στόχος, το νου σου ε…''\\%
\end{normalsize}%
\end{greek}\\[0.2cm]%
Katerina Gogou%


\chapter{Abstract}%
It is a popular practice to use materialized intermediate results to
improve performance of RDBMSes. Work in this area has focused either on
optimizers matching existing results or selecting useful intermediate
results from a plan, but few attempts have been made to create plans
with intermediate results in mind, and none that make any deduplicate
the stored data to alleviate the storage cost of maintaining possibly
large queries.

We built \emph{FluiDB} to explore a novel approach to integrating the
selection of materialized results with the planner in order to
optimize the logical representation of data in memory. FluiDB
materializes hot intermediate results and deduplucates data to
alleviate the cost of maintaining them. This is achieved by
introducing \emph{reversible operations}, versions of normal relational
operators that optionally pass enough data to the output to make the
input relations reconstructable. A planner aware of such operations
can build query plans that dynamically adapt the data layout to the
plan under constrained memory budget. This thesis revolves around four
main chapters each of which describes in detail a different part of
FluiDB and a final one that goes into evaluation of the system.

The first chapter focuses on query processing and the relational
algebra semantics that FluiDB operates under. FluiDB parses queries
into graphs of sub-queries connected by reversible operators. Each
such graph of the workload is merged into a global query graph that is
used to infer properties of each relation like cardinality and
extent.

The next chapter is dedicated to the planner and a novel monad for
weighted backtracking that the planer is based on. The planner
attempts to generate a plan based on the query graph that besides
solving the query at hand, leaves in memory an optimal set of queries
for the workload being run. In this chapter, the garbage collector is
also discussed, that creates deletion operators as part of the plan
such that the available budget is respected.

After that, we go into \emph{Antisthenis}, a novel incremental
computation system used by the query planner to determine, given a
particular set of materialized relations, whether a relation
materializable and the expected cost of materializing a
relation. Antisthenis, besides reusing computations, takes advantage
of the properties of the operations involved in the expression she is
evaluating, like absorbing group elements, to heuristically avoid as
much work as possible.

The final chapter about the FluiDB architecture describes the
transpilation of plans generated by the planner to C++, as well as the
supporting libraries that enable the evaluation of queries as C++
code, and the low level data organization of the database. The thesis
closes with a chapter that describes our methods for benchmarking and
some experimental results.


\chapter{Lay Summary}%
A lay summary.

\chapter{Acknowledgements}%

This thesis is dedicated to all the people that directly or indirectly
supported me though my PhD studies, but also to all the people
whithout whose support and hard work I would never have started this
work to begin with.

It is customary to thank the supervisors first regardless of the
circumstances but I would like to underline my gratitude to Stratis
who really understood my idiosyncratic way of working and provided
precisely the quality and quantity of guidance that I required. This
is a rare quality and I cherish the opportunity I had to work with
him. I would also like to express my gratitude to thank Boris Grot,
not just in the general sense but because, especially during the first
years of my thesis he went out of his way to make me feel included and
gave value to my voice. For the similar reasons I would like to thank
Christophe Dubach whose door was always open and his whiteboard always
available, I deeply appreciate all the times he was willng to
patiently listen to my crazy ideas.

I would like to thank my family and especially my parents, for being
with me every step of the bumpy road that was my PhD, but mostly
because without their hard work and sacrifises a PhD would not have
been an option for me. I am priviledged and humbled and I owe it
primarily to them.

I would also like to thank Marina without whom I would definitely have
given up in one of the many moments of desparation. Thank you for
helping me with the impossible task of finishing a PhD during a
pandemic.

Special thank you to my flatmate for two years Christos Velanis who
got me safe and sound through some fairly dark times, and for the
ideological steeling brought about by this cohabitation. Thanks to him
and to Christos Maniatis for the heated philosophical debates that
would typically end after 6am.

Last but not least I would like to thank my friends that were there
though all of it, the thick and the thin Antreas, Giannis, Aisha,
Maria, Irini, Iordanis, Nelly, Christos, Pigi, Niki, The President,
and Nikolas.

%%%% DECLARATION

%% Use a custom declaration

% \declaration{I did it.}

%% Use the standard regulation declaration. Enter your
%% name for the signature line.

\standarddeclaration{Christos Perivolaropoulos}

\end{precontent}


\tableofcontents

%% List of figures
\cleardoublepage
\phantomsection
\addcontentsline{toc}{chapter}{\listfigurename}
\listoffigures
\listoflistings

\cleardoublepage%
