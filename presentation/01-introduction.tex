\begin{frame}
\frametitle{FluiDB at a glance}

\begin{itemize}
\item FluiDB is an in-memory RDBMS that optimizes data layout for
  space efficiency w.r.t. the workload
\item The main novelty relates to the introduction of reversible
  relational operations which affords a new perspective on query
  planning and view selection.
\item FluiDB materializes all intermediate results and deletes garbage
  collects when she runs out of space.
\end{itemize}
\end{frame}


\newcommand{\n}[1]{node {\(#1\)}}
\newcommand{\bn}[1]{node[mat] {\(#1\)}}

\begin{frame}
  \frametitle{Fundamental principle}

  \begin{tikzdiagram_w}
    \tikzset{
      grow=up,
      sibling distance=3cm,
      level distance=3cm,
      every node/.style={draw},
      mat/.style={fill=gray!30}
    };

    \node {\(\li\)} % \li
    child { \bn{\sigma_{q} \li }} % σ_q
    child { \bn{\sigma_{\neg q} \li}} % σ_nq
    child { \n{\sigma_{\neg p} \li}}  % σ_np
    child {
      \n{\sigma_{p} \li} % σ_p
      child {\n{\sigma_{p \land r} \li} } % σ_pq
      child {\n{\sigma_{p \land \neg r} \li}} % σ_rpq
    } ;
  \end{tikzdiagram_w}

\end{frame}

\begin{frame}
  \frametitle{The interesting components}

  \begin{itemize}
  \item Graph management and query normal form representation
  \item Logical planning infrastructure
  \item Antisthenis: An incremental numeric evaluation system for cost
    estimation.
  \item Logical planning algorithm and garbage collector
  \item Code generation system.
  \end{itemize}
\end{frame}

\begin{frame}
  \frametitle{Architecture}


  % XXX: At the beginning of each case circle the relevant part of it
  % as an introduction.
  \begin{tikzdiagram_h}
    \tikzset{db/.style={cylinder,draw,shape border rotate=90,aspect=.3}};
    \tikzset{sys/.style={rectangle,draw}};
    \tikzset{outer/.style={cloud,draw,aspect=3}};

    \node[outer] (query) {Query stream};
    \node[sys] (graph_builder) [below = of query] {Graph Builder};

    \node[sys] (qnf) [left = of graph_builder] {QNF processor};

    \node[db] (bqg) [right = of graph_builder] {RA Graph};
    \node[sys] (planner) [below = of graph_builder] {Planner};

    \node[sys] (antisthenis) [left = of planner] {Antisthenis};
    \node[sys] (codegen) [below = of planner] {C++ generation};
    \node[sys] (cc) [below = of codegen] {C++ Compiler};
    \node[outer] (hw) [below = of cc] {CPU/Filesystem};

    \draw[->] (query) -> (graph_builder) -> (planner) -> (codegen) -> (cc) -> (hw);
    \draw[<->] (planner) -> (antisthenis);
    \draw[<->] (graph_builder) -> (qnf);
    \draw[<->] (graph_builder) -> (bqg);
  \end{tikzdiagram_h}
\end{frame}

%%% Local Variables:
%%% mode: latex
%%% TeX-master: "presentation"
%%% End:
