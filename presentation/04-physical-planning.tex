\begin{frame}
  \frametitle{Physical planning}
  \framesubtitle{\hask{HCntT} logic monad}

  Logic framework for ``fair'' traversal of the plan search space. Intricudes:

  \begin{itemize}
  \item \hask{a <//> b}: Try the rest of the computation with \hask{a}
    and if it fails try \hask{b}.
  \item \hask{once c}: try the continuation with values from \hask{c}
    until one works and stick with that one.
  \item \hask{halt n}: yield to a scheduler and assigne priority
    \hask{n} to the continuation.
  \end{itemize}
\end{frame}

\begin{frame}[fragile]
  \frametitle{Physical planning}
  \framesubtitle{Business logic}
  % XXX: Show in a tree.
  \begin{code}
    \begin{haskellcode}
    materialize n = unless (materialized n) $ do
      op <- inputOps n
      outputs <- possibleOutputs n op
      let inputs = inputsOf op
      -- Assuming we materialized the output, what is the cost of the
      -- outputs
      once (gc outputs)
      histCost <- withMaterialized outputs $ historicalCosts
      -- Stop and schedule this branch according to its cost
      halt (cost op + histCost + anticipatedCost inputs)
      -- Recursively materialize the input relations
      mapM materialize inputs
      registerPlan op
      mapM (setState Materialized) output
    \end{haskellcode}
  \end{code}
\end{frame}

%%% Local Variables:
%%% mode: latex
%%% TeX-master: "presentation"
%%% End:
