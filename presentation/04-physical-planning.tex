
\defverbatim[colored]\alternatives{
\begin{haskellcode}
  indexWords :: [(String,String)]
  indexWords = do
    sentence <- return "the red and brown fox" <|> return "the black and blue bear"
    w <- words sentence -- for each word
    guard $ w `notElem` ["the", "and"] -- skip the boring words
    return (w,sentence)
\end{haskellcode}
}

\defverbatim[colored]\alternativesres{
\begin{verbatim}
> indexWords
[("red","the red and brown fox")
,("brown","the red and brown fox")
,("fox","the red and brown fox")
,("black","the black and blue bear")
,("blue","the black and blue bear")
,("bear","the black and blue bear")]
\end{verbatim}
}

\begin{frame}{List as a monad = backtracking}
  \alternatives
  \uncover<2>{ \alternativesres}
\end{frame}


\defverbatim[colored]\unfairexample{
\begin{haskellcode}
  nonTerm :: [(Int,Int,Int)]
  nonTerm = do
    -- >  (,) <$> [1..3] <*> [1..3]
    -- [(1,1),(1,2),(1,3),(2,1),(2,2),(2,3),(3,1),(3,2),(3,3)]
    (a,b,c) <- (,,) <$> [0..] <*> [0..] <*> [..]
    guard $ a + b - c == 10
    return (a,b,c)
\end{haskellcode}
}

\defverbatim[colored]\unfairres{
\begin{verbatim}
> take 3 nonTerm
\end{verbatim}
}


\begin{frame}{List based logic/backtracking (unfair)}
  \unfairexample
  \uncover<2>{\unfairres}
  \uncover<2>{\centering \(\bot\)}
\end{frame}


\defverbatim[colored]\fairexample{
\begin{haskellcode}
  term :: [(Int,Int,Int)]
  term = do
    -- (>*<) :: [a -> b] -> [a] -> [b]
    -- > (,) <$> [1..3] >*< [1..3]
    -- [(1,1),(2,1),(1,2),(2,2),(3,1),(3,2),(1,3),(2,3),(3,3)]
    (a,b,c) <- (,,) <$> [0..] >*< [0..] >*< [0..]
    guard $ a + b - c == 10
    return (a,b,c)
\end{haskellcode}
}

\defverbatim[colored]\fairexampleres{
\begin{verbatim}
  > take 5 term
  [(5,5,0),(6,4,0),(6,5,1),(4,6,0),(5,6,1)]
\end{verbatim}
}

\begin{frame}{List based logic/backtracking (fair)}
\fairexample
\uncover<2>{\fairexampleres}
\end{frame}

\begin{frame}
  \frametitle{Physical planning}
  \framesubtitle{\hask{HCntT} logic monad}

  Logic framework for ``fair'' traversal of the plan search space. Intricudes:

  \begin{itemize}
  \item \hask{a <//> b}: Try the rest of the computation with \hask{a}
    and if it fails try \hask{b}.
  \item \hask{once c}: try the continuation with values from \hask{c}
    until one works and stick with that one.
  \item \hask{halt n}: yield to a scheduler and assigne priority
    \hask{n} to the continuation.
  \end{itemize}
\end{frame}

\defverbatim[colored]\gcexample{
  \begin{haskellcode}
    gc reqSize = do
      -- Try the current epoch and if that fails retry with a new epoch.
      return () <//> newEpoch
      -- find the deletable n-nodes and sort them by size
      deleteables <- sortOnM getNodeSize =<< filterM isDeletable =<< getAllNodes
      -- Try deleting each n-node and stop deleting when amassing enough
      -- free pages.
      forM_ deletables $ \n -> do
        freePgs <- getFreePages
        when (freePgs < reqSize) $ tryDelete n <//> markAsConcrete n
\end{haskellcode}
}

\begin{frame}{The GC}
  \gcexample
\end{frame}

\begin{frame}[fragile]
  \frametitle{Physical planning}
  \framesubtitle{Business logic}
  % XXX: Show in a tree.
  \begin{code}
    \begin{haskellcode}
    materialize n = unless (materialized n) $ do
      op <- inputOps n
      outputs <- possibleOutputs n op
      let inputs = inputsOf op
      -- Assuming we materialized the output, what is the cost of the
      -- outputs
      once (gc outputs)
      histCost <- withMaterialized outputs $ historicalCosts
      -- Stop and schedule this branch according to its cost
      halt (cost op + histCost + anticipatedCost inputs)
      -- Recursively materialize the input relations
      mapM materialize inputs
      registerPlan op
      mapM (setState Materialized) output
    \end{haskellcode}
  \end{code}
\end{frame}

%%% Local Variables:
%%% mode: latex
%%% TeX-master: "presentation"
%%% End:
