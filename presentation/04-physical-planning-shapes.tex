\newcommand\planner[9]{
  \begin{tikzdiagram_h}
    \tikzset{node distance=2cm};
    \tikzset{nnode/.style={ellipse,draw}};
    \tikzset{tnode/.style={rectangle,draw}};

    \node[nnode] (a) [fill=#1] {A};
    \node[nnode] (b) [right of=a,fill=#2] {B};
    \node[nnode] (c) [right of=b,fill=#3] {C};
    \node[tnode] (to1) [below of = b] {};
    \path (to1) edge (a);
    \path (to1) edge (b);
    \path (to1) edge (c);
    \node[nnode] (d) [below left of = to1,fill=#4] {X};
    \node[nnode] (e) [below right of = to1,fill=#5] {E};
    \path (to1) edge (e);
    \path (to1) edge (d);

    \node[tnode] (to2) [above left of = d] {};
    \node[nnode] (f) [above left of = to2,fill=#6] {F};
    \node[nnode] (g) [below left of = to2,fill=#7] {G};
    \path (to2) edge (d);
    \path (to2) edge (f);
    \path (to2) edge (g);

    \node[tnode] (tu1) [below of = d] {};
    \node[nnode] (h) [below left of = tu1,fill=#8] {H};
    \node[nnode] (i) [below right of = tu1,fill=#9] {I};
    \path (tu1) edge (d);
    \path (tu1) edge (h);
    \path (tu1) edge (i);
  \end{tikzdiagram_h}
}

\renewcommand\n{none}
\newcommand\g{green!30}
\newcommand\dg{green}
\renewcommand\o{gray!30}
\newcommand\oo{gray!20}
\newcommand\inputsetsmsg{For each (materializable) input set}

\begin{frame}{Planning algorithm}
  \only<1>{
    \framesubtitle{Materialize node X (if it is not already materialized)}
    \planner{\n}{\n}{\n}{\g}{\n}{\n}{\n}{\n}{\n}
  }
 \only<2>{
    \framesubtitle{\inputsetsmsg}
    \planner{\n}{\n}{\n}{\o}{\n}{\g}{\n}{\n}{\n}
  }
 \only<3>{
    \framesubtitle{\inputsetsmsg}
    \planner{\n}{\n}{\n}{\o}{\n}{\n}{\n}{\g}{\g}
  }
 \only<4>{
    \framesubtitle{\inputsetsmsg}
    \planner{\g}{\g}{\g}{\o}{\n}{\n}{\n}{\n}{\n}
  }
 \only<5>{
    \framesubtitle{For each output set containing the node}
    \planner{\oo}{\oo}{\oo}{\o}{\g}{\n}{\n}{\n}{\n}
  }
 \only<6>{
   \framesubtitle{Schedule the current branch}
   \[priority = \sum_{op \in \text{planned ops}} cost(op) + \sum_{h \in hist.} cost_{stoch.}(h) + \sum_{d \in deps} cost_{exp.}(d)\]
  }
 \only<7>{
   \framesubtitle{Recursively materialize the inputs}
   \planner{\g}{\g}{\g}{\o}{\n}{\n}{\n}{\n}{\n}
 }
 \only<8>{
   \framesubtitle{Once the inputs are materialized, garbage collect to make room for the outputs}
   \planner{\dg}{\dg}{\dg}{\g}{\g}{\n}{\n}{\n}{\n}
 }
 \only<9>{
   \framesubtitle{Mark the outputs as materialized and register the operator for the plan}
   \planner{\dg}{\dg}{\dg}{\dg}{\dg}{\n}{\n}{\n}{\n}
 }

\end{frame}

\begin{frame}{Profit!}
  \centering{\emoji{moneybag}}
\end{frame}

\begin{frame}{Planning algorithm (recap)}
  \begin{itemize}
  \item A network with reverse nodes
  \item Check depsets for materializable
  \item Chose an output set
  \item Halt by combining
    \begin{itemize}
    \item Cost of operations so far.
    \item Cost of historical costs given the materialized nodes.
    \item Expected cost of input.
    \end{itemize}
  \item Recurse on chosen inputs.
  \item Garbage collect to make space forthe output set
  \item Mark outputs as materialized and register the operator as tirggered.
  \end{itemize}
\end{frame}


%%% Local Variables:
%%% mode: latex
%%% TeX-master: "presentation"
%%% End:
